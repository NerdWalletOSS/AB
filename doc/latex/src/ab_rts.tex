\documentclass[letterpaper]{article}
\usepackage{times}
\usepackage{helvet}
\usepackage{courier}
\usepackage{fancyheadings}
\usepackage{hyperref}
\pagestyle{fancy}
\usepackage{pmc}
\usepackage{graphicx}
\usepackage{verbatim}
\setlength\textwidth{6.5in}
\setlength\textheight{8.5in}
\input{../styles/ramesh_abbreviations}
\begin{document}
\title{AB --- Design of the RTS}
\author{ Ramesh Subramonian }
\maketitle
\thispagestyle{fancy}
\lhead{}
\chead{}
\rhead{}
\lfoot{{\small Decision Sciences Team}}
\cfoot{}
\rfoot{{\small \thepage}}

\section{Introduction}

The AB Run Time Server serves two purposes 
\be
\item Its primary purposes is to provide bucketing decisions for AB tests for
  other services and applications. For example, should user A be treated as
  Control or as Variant for test X? 
\item It is also used by Marketing to route SEM (Search Engine Marketing)
  traffic 
  \ee

\section{Speed and Scalability}

Since the AB RTS is a blocking (see disclaimer below) call for most other
services, it must have a very low response time, otherwise it will adversely
impact user experience. Also, since experimentation is
wide spread throughout the site, it must have high throughput.
Our goal is to be able to support an average response time of 100
\(\mu\)-seconds while supporting a traffic of 10,000 requests per second.

Are calls to the AB RTS blocking? When used as an AB system, the clients are
programmed to timeout if the response time is excessive. In such cases, they
upstream systems behave ``as if'' the AB RTS had responded with ``Control''.
When used as a marketing traffic router, the AB RTS is blocking.

\section{The use of Lua and C}

\subsection{Naming conventions}

All Lua functions exposed to C will be prefixe with \verb+l_+, e.g.,
\verb+l_add_test()+
\section{Multi-threading}

For all practical purposes, the AB RTS is a single-threaded system.
However, in order to keep the main thread free to respond to client requests, we
have 2 additional threads for the following purposes
\be
\item {\bf Logging:} The RTS is required to send an HTTP POST request to a
  logging server, informing about it bucketing decisions that it makes. 
\item {\bf Updates:} Changes made in the AB Admin Interface need to be
  propagated to the RTS. These cause updates to the data structures that the RTS
  uses to respond quickly.
  \ee

\section{Global Variables}

Global variables are 
\bi
\item declared in \verb+ab_globals.h+
\item initialized in \verb+zero_globals()+
\item freed in \verb+zero_globals()+
\item all start with \verb+g_+
\item most of these are written to by Lua and read by C. There are a few
  exceptions, as noted in \verb+ab_globals+
\ei

\subsection{Counters and statd}

\section{Endpoints}
The endpoints listed in Table~\ref{tbl_endpoints} are supported.
\begin{table}[hb]
\centering
\begin{tabular}{|l||l|l|l|l|l|l|l|l|}  \hline \hline
  {\bf Endpoint } & {\bf Read/Write} & {\bf Responsible party} \\ \hline \hline
\input{_endpoints.tex}
\hline
\end{tabular}
\caption{End-points}
\label{tbl_endpoints}
\end{table}

\section{Logging bucketing decisions}

The AB RTS logs bucketing decisions by making a POST request to the logger
service with the fields listed below
\verbatiminput{_logger_fields}

\TBC
\end{document}
