\documentclass[letterpaper]{article}
\usepackage{times}
\usepackage{helvet}
\usepackage{courier}
\usepackage{fancyheadings}
\usepackage{hyperref}
\pagestyle{fancy}
\usepackage{pmc}
\usepackage{graphicx}
\setlength\textwidth{6.5in}
\setlength\textheight{8.5in}

%%%%%%%%%%%%%%%%%%%%%%%%%%%%%%%%%%%%%%%%%%%%%%%%%%%%%%%%%%%%%%%%%%%%%%%%%%%%%%%%%%
% Special for e-unibus doc commands

\newcommand{\ForLater}{
\begin{center}
{\bf NOT FOR CURRENT VERSION}
\end{center}
}
\newcommand{\TBC}{\framebox{\textbf{TO BE COMPLETED}}}
\newcommand{\DISCUSS}{\Ovalbox {\bf \textcolor{red}{FOR DISCUSSION}}}
\newcommand{\Input}{\framebox{\textsf{in}}}
\newcommand{\Output}{\framebox{\textsf{out}}}
\newcommand{\debug}[1]{\textbf{debug start} #1 \textbf{debug finish}}
\newcommand{\inx}[1]{\emph{#1}}
\newtheorem{notation}{Notation}
\newtheorem{definition}{Definition}
\newtheorem{problem_statement}{Problem Statement}
\newtheorem{invariant}{Invariant}
\newtheorem{assumption}{Assumption}
\newtheorem{resource_string}{Resource String}
\newtheorem{testcase}{Test Case}
\newtheorem{note}{Note}
\newtheorem{specification}{Specification}
\newtheorem{caution}{Caution}
\newtheorem{prereq}{Pre-requisite}
\newtheorem{action}{Action}
\newtheorem{query}{Query}
\newcommand{\beq}{\begin{equation}} %% new, no conflict
\newcommand{\eeq}{\end{equation}} %% new, no conflict
\newcommand{\be}{\begin{enumerate}}
\newcommand{\ee}{\end{enumerate}}
\newcommand{\bi}{\begin{itemize}}
\newcommand{\ei}{\end{itemize}}
\newcommand{\bv}{\begin{verbatim}}
\newcommand{\ev}{\end{verbatim}}
\newcommand{\bd}{\begin{description}}
\newcommand{\ed}{\end{description}}
\newcommand{\bpre}{\begin{prereq}}
\newcommand{\epre}{\end{prereq}}
\newcommand{\bact}{\begin{action}}
\newcommand{\eact}{\end{action}}
\newcommand{\bs}{\begin{specification}}
\newcommand{\es}{\end{specification}}
\newcommand{\btc}{\begin{testcase}}
\newcommand{\etc}{\end{testcase}}
\newcommand{\bc}{\begin{caution}}
\newcommand{\ec}{\end{caution}}
\newcommand{\la}{\leftarrow}
\newcommand{\IpArgs}{\subsection{Input Arguments}}
\newcommand{\PreReqs}{\subsection{Pre-requisities}}
\newcommand{\Actions}{\subsection{Actions}}
\newcommand{\Coverage}{{\bf To test coverage.}}

%%%%%%%%%%%%%%%%%%%%%%%%%%%%%%%%%%%%%%%%%%%%%%%%%%%%%%%%%%%%%%%%%%%%%%%%%%%


\newtheorem{theorem}{Theorem}[section]
\newtheorem{lemma}[theorem]{Lemma}
\newtheorem{proposition}[theorem]{Proposition}
\newtheorem{corollary}[theorem]{Corollary}

\newenvironment{proof}[1][Proof]{\begin{trivlist}
\item[\hskip \labelsep {\bfseries #1}]}{\end{trivlist}}
\newenvironment{intuition}[1][Intuition]{\begin{trivlist}
\item[\hskip \labelsep {\bfseries #1}]}{\end{trivlist}}
%% \newenvironment{definition}[1][Definition]{\begin{trivlist}
%% \item[\hskip \labelsep {\bfseries #1}]}{\end{trivlist}}
\newenvironment{example}[1][Example]{\begin{trivlist}
\item[\hskip \labelsep {\bfseries #1}]}{\end{trivlist}}
\newenvironment{remark}[1][Remark]{\begin{trivlist}
\item[\hskip \labelsep {\bfseries #1}]}{\end{trivlist}}

\newcommand{\qed}{\nobreak \ifvmode \relax \else
      \ifdim\lastskip<1.5em \hskip-\lastskip
      \hskip1.5em plus0em minus0.5em \fi \nobreak
      \vrule height0.75em width0.5em depth0.25em\fi}

%%%%%%%%%%%%%%%%%%%%%%%%%%%%%%%%%%%%%%%%%%%%%%%%%%%%%%%%%%%%%%%%%
% \newcommand{\Alogon}{\mbox{\fontfamily{ptm}\selectfont {\large \selectfont A} \hspace{-1.2ex} {\large \selectfont L} \hspace{-2.3ex} \raisebox{0.45ex}{ {\footnotesize \selectfont O} } \hspace{-1.80ex} {\large \selectfont G} \hspace{-1.80ex} \raisebox{-0.33ex}{ {\large \selectfont O} } \hspace{-1.8ex} {\large \selectfont N}}}


\begin{document}
\title{AB --- User Interface}
\author{ Ramesh Subramonian }
\maketitle
\thispagestyle{fancy}
\lhead{}
\chead{}
\rhead{}
\lfoot{{\small Data Science Team}}
\cfoot{}
\rfoot{{\small \thepage}}

\section{Introduction}

\section{UI Guidelines}
\subsection{Global Variables}
The UI maintains the following global variables. 
\be
\item Admin Name, Displayed at top of all pages
\item TestType, can be AB or XY (for URL Router), 
  Displayed at top of all pages
\ee

\subsection{Client side validation}
Client-side validation is considered a ``nice-to-have'' as long as it does not
overly complicate the code. All text-boxes will have some defaults for the
maximum number of characters they will accept.

\section{Login Page}

Set Admin Name and decide TestType

\section{Home Page}
\label{home_page}
Contains 
\be
\item a {\tt PLUS} button that leads to Section~\ref{add_edit_basic_page} 
  with GET param TestID as null
\item A radio button that allows one of the following options for State. 
  It restricts tests displayed to those that have the selected state(s)
  \be
\item Draft
\item Dormant, Started, Terminated --- default 
\item Archived
  \ee
\item A table of existing tests, Table~\ref{tbl_home}. We display all valid
  state/action pairs. Note that clicking on an Action causes the corresponding
  TestID to be sent as a GET parameter to the destination page.
\ee

\begin{table}[hb]
\centering
\begin{tabular}{|l||l|l|l|l|l|l|l|l|}  \hline \hline
  {\bf ID} & {\bf Name} & {\bf State} & {\bf Action} & {\bf Details} \\ \hline \hline
  5 & Test0 & Draft & {\tt Edit, Dormant, Delete} & Section~\ref{edit_test},
  ~\ref{del_test} \\ \hline 
  10 & Test1 & Dormant & {\tt Edit, Start, Delete} &  
  Section~\ref{edit_test}, ~\ref{start_test}, ~\ref{del_test} \\ \hline 
  20 & Test2 & Started & {\tt Edit, Terminate} & 
  Section~\ref{edit_test}, ~\ref{archive_test} \\ \hline 
  30 & Test3 & Terminated & {\tt View, Archive} & 
  Section~\ref{view_test}, ~\ref{archive_test} \\ \hline 
  40 & Test4 & Archived & {\tt View}  &  Section~\ref{edit_test} \\ \hline
\hline
\end{tabular}
\caption{Home Page}
\label{tbl_home}
\end{table}

%--------------------------------------------------------------

\section{Actions}
\subsection{Edit Test}
\label{edit_test}
Leads to Section~\ref{add_edit_basic_page}

\subsection{Dormant Test}
\label{dormant_test}
Verifies that test is in good shape, changes state to dormant and returns to home page Section~\ref{home_page}

\subsection{Start Test}
\label{start_test}
Changes state to start and returns to home page Section~\ref{home_page} after a
confirmation pop-up

\subsection{Stop Test}
\label{stop_test}
Leads to Section~\ref{add_edit_basic_page} with GET param {\tt action = terminate}

\subsection{Archive Test}
\label{archive_test}

Changes state to archived and returns to home page Section~\ref{home_page} after
a confirmation pop-up

\subsection{View Test}
\label{view_test}
Leads to Section~\ref{add_edit_basic_page}

\subsection{Delete Test}
\label{del_test}

Does a hard delete of the test after a confirmation pop-up
and returns to home page Section~\ref{home_page}


\section{Add/Edit Basic Page}
\label{add_edit_basic_page}
This page has the GET params TestID and Action. We derive ``mode'' using the
table Table~\ref{tbl_mode}
\begin{table}[hb]
\centering
\begin{tabular}{|l||l|l|l|l|l|l|l|l|}  \hline \hline
{\bf TestID} & {\bf State} & {\bf Mode} \\ \hline
Undefined     & --- & Add \\ \hline
Defined & Dormant, Draft & Edit \\ \hline
Defined & Started & View (except Percentage can be edited)\\ \hline
Defined & Terminated, Archived & View \\ \hline
\hline
\end{tabular}
\caption{Mode for Add/Edit Basic Page}
\label{tbl_mode}
\end{table}

%---------------------------------------------
The page has the following UI elements
\be
\item Text box for test name 
\item Text box for test description
\item Variant information consists of a table as in Table~\ref{tbl_page_1}
\item A {\bf Next} button, leading to Section~\ref{addnl_variant_info}
  \ee
\begin{table}[hb]
\centering
\begin{tabular}{|l||l|l|l|l|l|l|l|l|}  \hline \hline
  {\bf Number} &   {\bf ID} & {\bf Name} & {\bf Percentage} 
  & {\bf URL} & {\bf Winner} \\ \hline \hline
  1 & --- & Control & 40   & \url{www.google.com} & N \\ \hline
  2 & --- & Variant A & 30  & \url{www.yahoo.com} & Y \\ \hline
  3 & --- & Variant B & 20  & \url{www.cnn.com} & N \\ \hline
\hline
\end{tabular}
\caption{Add/Edit Page}
\label{tbl_page_1}
\end{table}
If Mode = View, nothing is editable. 
Depending on Mode being Add or Edit, this page behaves somewhat differently as shown below.
 However, in both of following cases, 
 (i)  Test Description is editable text box
 and (ii) Percentage is editable text box
 \subsection{Add Mode}
 \label{add_edit_basic_page_add_mode}
 \bi
 \item Test Name is editable test box 
 \item ID is not viewable
 \item Name is editable text box
 \item URL is editable text box
 \item Winner is not viewable
\ei
 \subsection{Edit Mode}
 \label{add_edit_basic_page_edit_mode}
 \bi
 \item Test Name is read-only
 \item ID is displayed as read-only
 \item Name is read-only
 \item URL is displayed as read-only 
 \item Winner is a radio button so only one row can be selected
   \ei

A few other notes
\bi
\item If TestType = AB, then the name of the first variant must be set to
  Control and is {\bf never} editable.
\item If Action = terminate, then an additional column called ``Winner'' is
displayed with radio buttons allowing the user to select any one variant as
winner. In this case, {\tt Next} button leads to home page, after a confirmation
pop-up.
\item URL.  Should be displayed only if TestType = XY
\ei
%------------------------------------------------

\section{Additional Variant Information}
\label{addnl_variant_info}
Looks exactly like Section~\ref{add_edit_basic_page}
with constraints of Section~\ref{add_edit_basic_page_edit_mode} with differences as follows
\be
\item {\tt Next} leads to Section~\ref{filters} if TestType = AB. 
Leads to Section~\ref{device_specific} if TestType = XY. 
\item Each row has
an {\tt Edit} button that creates a pop-up that allows user to view/edit 3 text
fields
\be
\item Custom Data.
  Should be displayed only if TestType = XY
\item Description
\ee
\ee

\section{Filters}
\label{filters}
This page is for use only when TestType = AB and state = draft or dormant
\be
\item {\bf Previous} leads to Section~\ref{addnl_variant_info}
\item {\bf Next} leads to Section~\ref{home_page}
\item Table as in Section~\ref{tbl_filters}
  \ee
\begin{table}[hb]
\centering
\begin{tabular}{|l|l|l|l|l|l|l|l|l|}  \hline \hline
  {\bf Attribute} &   {\bf Value} & {\bf Checkbox} \\ \hline \hline
  Platform & Desktop & \\ \hline
  Platform & Tablet & \\ \hline
  Platform & Mobile & \\ \hline
  \hline
  IsPaid & true & \\ \hline
  IsPaid & false & \\ \hline
\hline
\end{tabular}
\caption{Setting Filters}
\label{tbl_filters}
\end{table}

\section{Device Specific Routing}
\label{device_specific}
This page is for use only when TestType = XY and State is draft, dormant,
started. 
\be
\item There is a check-box which indicates whether 
  \verb+is_device_specific+ is true for this test. This can be turned on and off
  at any point in time. Turning it off means that we use the percentage
  information in Section~\ref{add_edit_basic_page}. Turning it on means that we use the
  percentage information in this page.
\item {\bf Previous} leads to Section~\ref{addnl_variant_info}
\item {\bf Next} leads to Section~\ref{follow_on}
\item Table as in Section~\ref{tbl_device_specific}
  \ee
\begin{table}[hb]
\centering
\begin{tabular}{|l||l|l|l|l|l|l|l|l|}  \hline \hline
  {\bf Variant Name} & 
  {\bf Desktop} & 
  {\bf Mobile iOS}  &
  {\bf Mobile Android}  & 
  {\bf Tablet iOS}  & 
  Tablet {\bf Android}  &
  Other \\ \hline \hline
  V1 & 100  & 0   & 20 & 40 & 60 & 80 \\ \hline
  V2 & 0    & 100 &  80 & 60 & 40 & 20 \\ \hline
\hline
\end{tabular}
\caption{Device Specific Routing}
\label{tbl_device_specific}
\end{table}

\section{Follow On}
\label{follow_on}
This page is for use only when TestType = XY and State is draft or dormant.
\be
\item Display the Channel of the test
\item {\bf Clear}, which keeps us in same page but clears any radio button
  selection
\item {\bf Previous} leads to Section~\ref{device_specific}
\item {\bf Next} leads to Section~\ref{home_page}
\item If there are no candidates to follow, then message indicating {\bf No
  Tests to Follow}; else, table as in Table~\ref{tbl_follow_on}
  \ee

\begin{table}[hb]
\centering
\begin{tabular}{|l||l|l|l|l|l|l|l|l|}  \hline \hline
  {\bf Test ID } & {\bf Test Name} & {\bf Radio Button}\\ \hline 
  10 & Test1 & \\ \hline
  20 & Test2 & \\ \hline
\hline
\end{tabular}
\caption{Selecting a Follow On}
\label{tbl_follow_on}
\end{table}

\section{Config}
We will have a ``hidden'' page for editing config variables as in
Table~\ref{tbl_config}

\begin{table}[hb]
\centering
\begin{tabular}{|l||l|l|l|l|l|l|l|l|}  \hline \hline
  {\bf Name } & {\bf Value} \\ \hline \hline
  \verb+default_landing_page+ & \url{www.nerdwallet.com} \\ \hline
\verb+check_url_reachable+ & true \\ \hline
  \verb+num_retries+ & 10 \\ \hline 
\hline
\end{tabular}
\caption{View/Edit Config Variables}
\label{tbl_config}
\end{table}

\end{document}
