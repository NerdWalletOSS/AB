%%_START_DELETE_FOR_BOOK
\documentclass[letterpaper]{article}
\usepackage{times}
\usepackage{helvet}
\usepackage{courier}
\usepackage{fancyheadings}
\usepackage{hyperref}
\pagestyle{fancy}
\usepackage{pmc}
\usepackage{graphicx}
\setlength\textwidth{6.5in}
\setlength\textheight{8.5in}

%%%%%%%%%%%%%%%%%%%%%%%%%%%%%%%%%%%%%%%%%%%%%%%%%%%%%%%%%%%%%%%%%%%%%%%%%%%%%%%%%%
% Special for e-unibus doc commands

\newcommand{\ForLater}{
\begin{center}
{\bf NOT FOR CURRENT VERSION}
\end{center}
}
\newcommand{\TBC}{\framebox{\textbf{TO BE COMPLETED}}}
\newcommand{\DISCUSS}{\Ovalbox {\bf \textcolor{red}{FOR DISCUSSION}}}
\newcommand{\Input}{\framebox{\textsf{in}}}
\newcommand{\Output}{\framebox{\textsf{out}}}
\newcommand{\debug}[1]{\textbf{debug start} #1 \textbf{debug finish}}
\newcommand{\inx}[1]{\emph{#1}}
\newtheorem{notation}{Notation}
\newtheorem{definition}{Definition}
\newtheorem{problem_statement}{Problem Statement}
\newtheorem{invariant}{Invariant}
\newtheorem{assumption}{Assumption}
\newtheorem{resource_string}{Resource String}
\newtheorem{testcase}{Test Case}
\newtheorem{note}{Note}
\newtheorem{specification}{Specification}
\newtheorem{caution}{Caution}
\newtheorem{prereq}{Pre-requisite}
\newtheorem{action}{Action}
\newtheorem{query}{Query}
\newcommand{\beq}{\begin{equation}} %% new, no conflict
\newcommand{\eeq}{\end{equation}} %% new, no conflict
\newcommand{\be}{\begin{enumerate}}
\newcommand{\ee}{\end{enumerate}}
\newcommand{\bi}{\begin{itemize}}
\newcommand{\ei}{\end{itemize}}
\newcommand{\bv}{\begin{verbatim}}
\newcommand{\ev}{\end{verbatim}}
\newcommand{\bd}{\begin{description}}
\newcommand{\ed}{\end{description}}
\newcommand{\bpre}{\begin{prereq}}
\newcommand{\epre}{\end{prereq}}
\newcommand{\bact}{\begin{action}}
\newcommand{\eact}{\end{action}}
\newcommand{\bs}{\begin{specification}}
\newcommand{\es}{\end{specification}}
\newcommand{\btc}{\begin{testcase}}
\newcommand{\etc}{\end{testcase}}
\newcommand{\bc}{\begin{caution}}
\newcommand{\ec}{\end{caution}}
\newcommand{\la}{\leftarrow}
\newcommand{\IpArgs}{\subsection{Input Arguments}}
\newcommand{\PreReqs}{\subsection{Pre-requisities}}
\newcommand{\Actions}{\subsection{Actions}}
\newcommand{\Coverage}{{\bf To test coverage.}}

%%%%%%%%%%%%%%%%%%%%%%%%%%%%%%%%%%%%%%%%%%%%%%%%%%%%%%%%%%%%%%%%%%%%%%%%%%%


\newtheorem{theorem}{Theorem}[section]
\newtheorem{lemma}[theorem]{Lemma}
\newtheorem{proposition}[theorem]{Proposition}
\newtheorem{corollary}[theorem]{Corollary}

\newenvironment{proof}[1][Proof]{\begin{trivlist}
\item[\hskip \labelsep {\bfseries #1}]}{\end{trivlist}}
\newenvironment{intuition}[1][Intuition]{\begin{trivlist}
\item[\hskip \labelsep {\bfseries #1}]}{\end{trivlist}}
%% \newenvironment{definition}[1][Definition]{\begin{trivlist}
%% \item[\hskip \labelsep {\bfseries #1}]}{\end{trivlist}}
\newenvironment{example}[1][Example]{\begin{trivlist}
\item[\hskip \labelsep {\bfseries #1}]}{\end{trivlist}}
\newenvironment{remark}[1][Remark]{\begin{trivlist}
\item[\hskip \labelsep {\bfseries #1}]}{\end{trivlist}}

\newcommand{\qed}{\nobreak \ifvmode \relax \else
      \ifdim\lastskip<1.5em \hskip-\lastskip
      \hskip1.5em plus0em minus0.5em \fi \nobreak
      \vrule height0.75em width0.5em depth0.25em\fi}

%%%%%%%%%%%%%%%%%%%%%%%%%%%%%%%%%%%%%%%%%%%%%%%%%%%%%%%%%%%%%%%%%
% \newcommand{\Alogon}{\mbox{\fontfamily{ptm}\selectfont {\large \selectfont A} \hspace{-1.2ex} {\large \selectfont L} \hspace{-2.3ex} \raisebox{0.45ex}{ {\footnotesize \selectfont O} } \hspace{-1.80ex} {\large \selectfont G} \hspace{-1.80ex} \raisebox{-0.33ex}{ {\large \selectfont O} } \hspace{-1.8ex} {\large \selectfont N}}}


\begin{document}
\title{AB --- The Database}
\author{ Ramesh Subramonian }
\maketitle
\thispagestyle{fancy}
\lhead{}
\chead{}
\rhead{}
\lfoot{{\small Data Science Team}}
\cfoot{}
\rfoot{{\small \thepage}}
%%_STOP_DELETE_FOR_BOOK
%%_START_ADD_FOR_BOOK
%%C \chapter{Database Design and Usage}
%%_STOP_ADD_FOR_BOOK
\section{Introduction}

The AB Admin Webapp stores information needed by the AB system in a relational
database. In this document, we describe how the design and usage of this
database. 
We have tried to hew as closely as possible to the 
naming conventions in
\url{http://guides.rubyonrails.org/active_record_basics.html}

Some more Rails conventions
\bi
\item  tables are plural, e.g., users, page\_views, clicks
\item primary key is always {\tt id}
\item foreign keys are primary key table singular concatenated with \verb+_id+
  e.g., fk in clicks pointing to users is \verb+user_id+, 
  fk in clicks to \verb+page_views+ is \verb+page_view_id+
\item Almost every table has cols ``created\_at'' and
``updated\_at'' of type datetime. 
\item Use suffix \verb+_on+ for dates, e.g., \verb+expires_on+ '2018-12-31'
(as opposed to \verb+_at+ used for datetime/timestamps).
\ei

There are four kinds of tables 
\be
\item Configuration tables. These do not directly contain information about AB test.
Instead, they contain auxiliary information e.g., list of authorized users,
kinds of tests supported, names od devices recognized, \ldots
\item Test info tables. These contain the details of tests created by users
\item Debugging tables. These are used to create audit reports and to
  analyze usage of the system. See Section~\ref{logging_tables}
\item Journaling tables. These contain changes made to the test info tables as
described in Section~\ref{journaling_tables}
\ee


\begin{table}[hb]
\centering
\begin{tabular}{|l||l|l|l|l|l|l|l|l|}  \hline \hline
  {\bf Name } & {\bf Table Type} \\ \hline \hline
\input{_table_type}
\hline
\end{tabular}
\caption{View/Edit Config Variables}
\label{tbl_config}
\end{table}

\section{Logging Tables}
\label{logging_tables}

Every request to the webapp causes a row to be inserted into
\verb+request_webapp+. The primary key of this row is then used in the
journaling tables. After the processing of the request is completed, the row in
question is updated with an (optional) error message, (optional) output message and status code.

\section{Journaling Tables}
\label{journaling_tables}

All changes to the database are made by the Admin WebApp using transactions.
There are no deletes to the database, only inserts and updates. 
In every transaction, a copy of the rows modified/created are stored in log
tables that are identical to the original table with the following exceptions
\be
\item there is no primary key
\item there are no constraints place on data in these tables
\item every field in the original table exists in the journaling table
\item there are two additional fields
  \be
\item \verb+request_webapp_id+ (Section~\ref{logging_tables}) used to 
  \be
\item tie together multiple rows affected by a particular transaction
\item corrrelate changes to the database with requests made to the Webapp
  \ee
\item \verb+txn_id+ which identifies the operation made by the middle-tier i.e.,
  it identifies the logical purpose for performing the transaction. This is a
  foreign key to the table \verb+txn_type+
  \ee
  \ee
For a ``hello world'' explanation of what this is about, see 
\verb+sql/triggers_example.sql+

\section{Permissions}

Hewing to the principle of least privelege, we create database users with
minimal ability to modify the database.  The two users are (1) {\tt webapp} (2)
{\tt backend}. All users can read all tables.

\be
\item Triggers allow inserts (no updates or deletes) into tables marked as
  ``journaling''
\item 
Neither webapp nor backend user can delete from any table. All 
deletes are soft deletes using the {\tt is\_del} field. 
\item 
Only inserts (no updates or deletes) are allowed into tables marked as
``debugging'' 
\item 
Webapp users can update and insert only into tables marked as ``test info''
\item 
  \ee





\end{document}

