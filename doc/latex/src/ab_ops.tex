%%_START_DELETE_FOR_BOOK
\documentclass[letterpaper]{article}
\usepackage{times}
\usepackage{helvet}
\usepackage{courier}
\usepackage{fancyheadings}
\usepackage{hyperref}
\pagestyle{fancy}
\usepackage{pmc}
\usepackage{graphicx}
\setlength\textwidth{6.5in}
\setlength\textheight{8.5in}
\input{../styles/ramesh_abbreviations}
\begin{document}
\title{AB --- Operations Manual}
\author{ Ramesh Subramonian }
\maketitle
\thispagestyle{fancy}
\lhead{}
\chead{}
\rhead{}
\lfoot{{\small Data Science Team}}
\cfoot{}
\rfoot{{\small \thepage}}
%%_STOP_DELETE_FOR_BOOK
%%_START_ADD_FOR_BOOK
%%C \chapter{Database Design and Usage}
%%_STOP_ADD_FOR_BOOK
\section{Introduction}

\section{WebApp}

The WebApp is written in PHP and (for testing ) deployed using Apache 
as the web server. 

\subsection{Building}

\subsubsection{Local Testing}
In order to facilitate quick testing, we provide
\begin{verbatim}
cd AB
bash install.sh 
\end{verbatim}
where the directory contains relevant zip files to install Apache 
and PHP from source.
This script also installs some packages using {\tt sudo apt-get update}

\subsubsection{For deployment}

\subsection{Configuration}

The webapp assumes that the config file is located in
\verb+/opt/abadmin/db2.json+ 
A sample is provided in \verb+AB/php/db.conf.json+

\subsubsection{Access to other servers}

The WebApp server needs access to the following servers. 
\be
\item {\tt RTS Finder} When this server is hit with the endpoint
  {\tt/DescribeInstances}, it returns a JSON array of the RTS's available. For
  each RTS, it returns the server and the port.
\item {\tt RTS} For testing purposes, it is often useful to bypass the Finder
  and hit a single RTS directly. Note that this is {\bf not} meant for a
  ``real'' deploy.
\item {\tt MySQL} The WebApp stores all its information in a MySQL database.
  Necessary credentials need to be provided.
  \ee

\subsection{Diagnostics}

\subsubsection{Consistency}

On \url{localhost:8080/AB/app/diagnostics/diagnostics.php}, one can check
\be
\item the configurations used
\item the RTS' available
  \ee

On \url{localhost:8080/AB/app/admin/admin.php}, one can check
\be
\item whether the database and the RTS are consistent
\item whether the database is self-consistent \TBC
\ee 

\subsubsection{Health check}
When deployed locally, this URL serves as a health check
\url{localhost:8080/AB/health_check.html}

\section{Run Time Server (RTS)}

\subsection{Operating System and Packages}
We have tested this with the most recent version of the Ubuntu server. 
We expect the following packages to have been installed
\be
\item gcc
\item g++
\item cmake
\ee

\subsection{Environment Variables}

The RTS expects the following environment variables to be set
\be
\item \verb+LUA_PATH+ 
  See \verb+AB/to_source+ for an example
\item \verb+LD_LIBRARY_PATH+
  See \verb+AB/src/to_source+ for an example
  \ee

\subsection{Building}

\subsubsection{Local Testing}
In order to facilitate quick testing, we support the following
\begin{verbatim}
cd AB/src/
make clean
make
\end{verbatim}

\subsubsection{For deployment}

This is where aio.sh comes in. Indrajeet \TBC

\subsection{Configuration}
A sample config file is provided in 
\verb+AB/src/ab.conf.json+ Using this as a template, the following need to be
configured. Note that if there is an error in loading the config file, the
server will {\bf not} start up. You need to ensure that entries are set
correctly and that servers/files/directories are where you say they will be.

When testing, the server is started as follows \verb+ ./ab_httpd ab.conf.json+

\subsubsection{Access to other servers}

The run time server needs access to the following servers. 
\be
\item MySQL --- to get to the database where AB information is stored. This ia
  {\bf mandatory}.
\item Logger --- to record actions taken by the AB server. This is optional.
\item Kafka --- to record actions taken by the AB server. This is optional.
\item StatsD --- to send UDP packets to DataDog. This is optional.
\ee

\subsection{APIs for debugging}

\be
\item {\tt /api/v1/health\_check} --- returns 200 to say server is alive
\item {\tt /CheckLoggerConnectivity} --- sends a fake payload to Logger and
  verifies that it gets a timely response
\item {\tt /CheckDBConnectivity} --- makes a simple request to the MySQL
  database and verifies that it gets a timely response
\item {\tt /CheckKafkaConnectivity} --- sends a fake payload to Kafka and
  verifies that it gets a timely response
\item {\tt /Diagnostics} --- checks the internal consistency of the RTS
\item {\tt /DumpLog} --- provides statistics of how RTS is doing in terms of
  both activity that it has done, errors it has encountered, response time,
  \ldots
\item {\tt /GetConfig} --- returns configs that have been provided to the
  RTS, with passwords redacted
\item {\tt /PingServer?Service=XXXX} where XXXX can be one of 
  \be
\item logger
\item webapp
\item kafka 
\item statsd 
  \ee
  This simply pings the health check URL on the server, if one has been
  provided, to verify that it is alive
  \ee
%------------------------------------

\end{document}
