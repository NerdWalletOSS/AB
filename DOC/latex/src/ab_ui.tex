\documentclass[letterpaper]{article}
\usepackage{times}
\usepackage{helvet}
\usepackage{courier}
\usepackage{fancyheadings}
\usepackage{hyperref}
\pagestyle{fancy}
\usepackage{pmc}
\usepackage{graphicx}
\setlength\textwidth{6.5in}
\setlength\textheight{8.5in}
\input{../styles/ramesh_abbreviations}
\begin{document}
\title{AB --- User Interface}
\author{ Ramesh Subramonian }
\maketitle
\thispagestyle{fancy}
\lhead{}
\chead{}
\rhead{}
\lfoot{{\small Decision Sciences Team}}
\cfoot{}
\rfoot{{\small \thepage}}

\section{Introduction}

\section{Global Variables}
The UI maintains the following global variables. 
\be
\item Admin Name, Displayed at top of all pages
\item TestType, can be AB or XY (for URL Router), 
  Displayed at top of all pages
\item Mode, can be Add, Edit, View
\item State of test if Mode = Edit
\item IsDeviceSpecific, boolean, default false
\ee

\section{Page 0}

Decide TestType

\section{Home Page}
\label{home_page}
Contains 
\be
\item a {\tt PLUS} button that leads to Section~\ref{add_edit_page} with {\tt mode =
  add}
\item A radio button that allows one of the following options for State
  \be
\item Draft
\item Dormant, Started, Terminated
\item Archived
  \ee
\item A table of existing tests, Table~\ref{tbl_home}. We display all valid
  state/action pairs
\ee

\begin{table}[hb]
\centering
\begin{tabular}{|l||l|l|l|l|l|l|l|l|}  \hline \hline
  {\bf ID} & {\bf Name} & {\bf State} & {\bf Action} & {\bf Details} \\ \hline \hline
  5 & Test0 & Draft & {\tt Edit, Delete} & Section~\ref{edit_test},
  ~\ref{del_test} \\ \hline 
  10 & Test1 & Dormant & {\tt Start, Delete} &  Section~\ref{start_test},
  ~\ref{del_test} \\ \hline 
  20 & Test2 & Started & {\tt Terminate, Archive} & Section~\ref{stop_test} \\ \hline 
  30 & Test3 & Terminated & {\tt Archive} & Section~\ref{archive_test} \\ \hline 
  40 & Test4 & Archived & --- &  \\ \hline
\hline
\end{tabular}
\caption{Home Page}
\label{tbl_home}
\end{table}

%--------------------------------------------------------------

\section{Navigations}
We will navigate between pages as in Table~\ref{tbl_navigations}
\begin{table}[hb]
\centering
\begin{tabular}{|l||l|l|l|l|l|l|l|l|}  \hline \hline
  {\bf Page} & {\bf Next} & {\bf Previous} & {\bf Globals} \\ \hline
  --- & --- & --- & --- \\ \hline
\hline
\end{tabular}
\caption{Navigations}
\label{tbl_navigations}
\end{table}

\section{Actions}
\subsection{Edit Test}
\label{edit_test}
Leads to Section~\ref{add_edit_page} with {\tt mode = edit}
\subsection{Start Test}
\label{start_test}
\subsection{Stop Test}
\label{stop_test}
\subsection{Archive Test}
\label{archive_test}

\section{Add/Edit Page}
\label{add_edit_page}
\be
\item Text box for test name 
\item Variant information consists of a table as in Table~\ref{tbl_page1}
\item A {\bf Next} button, leading to Section~\ref{addnl_variant_info}
\item A {\bf Home} button, leading to Section~\ref{home_page}
  \ee
\begin{table}[hb]
\centering
\begin{tabular}{|l||l|l|l|l|l|l|l|l|}  \hline \hline
  {\bf Number} &   {\bf ID} & {\bf Name} & {\bf Percentage} \\ \hline \hline
  1 & --- & Control & 40  \\ \hline
  2 & --- & Variant A & 30  \\ \hline
  3 & --- & Variant B & 20  \\ \hline
\hline
\end{tabular}
\caption{Add/Edit Page}
\label{tbl_page_1}
\end{table}
 Depending on mode, this page behaves somewhat differently
 \subsection{Add Mode}
 \label{page1_add_mode}
 \bi
 \item Test Name is editable test box 
 \item ID column is NOT displayed
 \item Name is editable text box
 \item Percentage is editable text box
   \ei
 \subsection{Edit Mode}
 \label{page1_edit_mode}
 \bi
 \item Test Name is read-only
 \item ID column is displayed
 \item Names is read-only
 \item Percentage is editable text box
   \ei

If IsDeviceSpecific = true, then percentgae column is not displayed.
\section{Additional Variant Information}
\label{addnl_variant_info}
Looks exactly like Section~\ref{page1}
with constraints of Section~\ref{page1_edit_mode} with differences as follows
\be
\item {\tt Next} leads to Section~\ref{XX}
\item Each row has
an {\tt Edit} button that creates a pop-up that allows user to view/edit 3 text
fields
\be
\item Custom Data
\item URL. Should not be editable if mode = edit and state = dormant
\item Description
\ee
\ee

\section{Filters}
\label{filters}
This page is for use only when TestType = AB and state = draft
\be
\item {\bf Previous} leads to Section~\ref{XX}
\item {\bf Next} leads to Section~\ref{XX}
\item Table as in Section~\ref{tbl_filters}
  \ee
\begin{table}[hb]
\centering
\begin{tabular}{|l|l|l|l|l|l|l|l|l|}  \hline \hline
  {\bf Attribute} &   {\bf Value} & {\bf Checkbox} \\ \hline \hline
  Platform & Desktop & \\ \hline
  Platform & Tablet & \\ \hline
  Platform & Mobile & \\ \hline
  \hline
  IsPaid & true & \\ \hline
  IsPaid & false & \\ \hline
\hline
\end{tabular}
\caption{Setting Filters}
\label{tbl_filters}
\end{table}

\section{Device Specific Routing}
\label{filters}
This page is for use only when TestType = XY and State is draft, dormant,
started
\be
\item {\bf Previous} leads to Section~\ref{XX}
\item {\bf Next} leads to Section~\ref{XX}
\item Table as in Section~\ref{tbl_device_specific}
  \ee
\begin{table}[hb]
\centering
\begin{tabular}{|l||l|l|l|l|l|l|l|l|}  \hline \hline
  {\bf Variant Name} & 
  {\bf Desktop} & 
  {\bf Mobile iOS}  &
  {\bf Mobile Android}  & 
  {\bf Tablet iOS}  & 
  Tablet {\bf Android}  &
  Other \\ \hline \hline
  V1 & 100  & 0   & 20 & 40 & 60 & 80 \\ \hline
  V2 & 0    & 100 &  80 & 60 & 40 & 20 \\ \hline
\hline
\end{tabular}
\caption{Device Specific Routing}
\label{tbl_device_specific}
\end{table}

\section{Follow On}
This page is for use only when TestType = XY and State is draft.
\be
\item {\bf Clear}, which keeps us in same page but clears any radio button
  selection
\item {\bf Previous} leads to Section~\ref{XX}
\item {\bf Next} leads to Section~\ref{XX}
\item If there are no candidates to follow, then message indicating {\bf No
  Tests to Follow}; else, table as in Table~\ref{tbl_follow_on}
  \ee

\begin{table}[hb]
\centering
\begin{tabular}{|l||l|l|l|l|l|l|l|l|}  \hline \hline
  {\bf Test ID } & {\bf Test Name} & {\bf Radio Button}\\ \hline 
  10 & Test1 & \\ \hline
  20 & Test2 & \\ \hline
\hline
\end{tabular}
\caption{Selecting a Follow On}
\label{tbl_follow_on}
\end{table}


\section{Delete Test}
\label{del_test}

Does a hard delete of the test 


\end{document}
