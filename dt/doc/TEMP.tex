% -*- latex -*-

\documentclass[12pt]{report}

% -*- latex -*-


\usepackage[left=1in, right=1in]{geometry}
\usepackage{times}
\usepackage[pdftex]{graphicx}
\usepackage{fancyhdr}
\usepackage{fancybox}
\usepackage{pstricks}
\usepackage{color}
\usepackage{sectsty}
\usepackage{makeidx}
\usepackage{tabularx}
\usepackage{ifthen}
\usepackage[pdftex,colorlinks,linkcolor=blue]{hyperref}
\usepackage{boxedminipage}
%% BELOW FROM RAMESH
\usepackage{supertabular}
\usepackage{verbatim}
\usepackage{ifthen}
\usepackage{tabularx}
\usepackage{afterpage}
\usepackage{soul}
\usepackage{amssymb}
\usepackage{amsmath}
%% ABOVE FROM RAMESH

% The main configuration file


\newcommand{\docversion}{0.1}
\newcommand{\docpublishingdate}{\today}



\makeindex

%%%%%%%%%%%%%%%%%%%%%%%%%%%%%%%%%%%%%%%%%%%%%%%%%%%%%%%%%%%%%%%%%%%%%%%%%%%%%%%%%%
% Special types of chapters
\newcommand{\specialchapter}[1]{\chapter*{#1}\markboth{#1}{}\addcontentsline{toc}{chapter}{#1}}

\newcommand{\acknowledgements}{\specialchapter{Acknowledgements}}
\newcommand{\introduction}{\specialchapter{Introduction}}
\newcommand{\preface}{\specialchapter{Preface}}
  \newcommand{\prefaceauthor}[1]{{\par\itshape \raggedleft    #1 \par}}
  \newcommand{\prefacewhere}[1]{{\par\itshape \raggedleft #1 \par}}
\newcommand{\acronyms}{\specialchapter{Acronyms}}
  \newcommand{\acro}[2]{\textsf{#1} --- #2 \par}
%%%%%%%%%%%%%%%%%%%%%%%%%%%%%%%%%%%%%%%%%%%%%%%%%%%%%%%%%%%%%%%%%%%%%%%%%%%%%%%%%%


%%%%%%%%%%%%%%%%%%%%%%%%%%%%%%%%%%%%%%%%%%%%%%%%%%%%%%%%%%%%%%%%%%%%%%%%%%%%%%%%%%
% Page and headings layout
\pagestyle{fancy}

\addtolength{\headwidth}{\marginparwidth}

\newcommand{\myhf}{
  \fancyhead{}
  \renewcommand{\headrulewidth}{1pt}
  \renewcommand{\footrulewidth}{1pt}

%  \fancyhead[R]{\rightmark{}\rule{1cm}{0pt}\includegraphics{logo}}
%  \fancyhead[R]{\rightmark{}{\hspace{0.2ex} \Ovalbox{LinkedIn}}}
  \fancyhead[L]{\leftmark{}}
%  \fancyhead[C]{v. \docversion{}, \docpublishingdate{}}
  \fancyhead[C]{\docpublishingdate{}}
%  \fancyfoot[L]{LinkedIn Proprietary and Confidential}
}

\myhf
\fancypagestyle{plain}{\myhf}

\renewcommand{\chaptermark}[1]{\markboth{#1}{}}
\renewcommand{\sectionmark}[1]{\markright{#1}}

\allsectionsfont{\sffamily}
\chapterfont{\raggedleft\sffamily\Huge}
%%%%%%%%%%%%%%%%%%%%%%%%%%%%%%%%%%%%%%%%%%%%%%%%%%%%%%%%%%%%%%%%%%%%%%%%%%%%%%%%%%

%% BY RAMESH
\setcounter{secnumdepth}{4}
%% BY RAMESH


%%%%%%%%%%%%%%%%%%%%%%%%%%%%%%%%%%%%%%%%%%%%%%%%%%%%%%%%%%%%%%%%%%%%%%%%%%%%%%%%%%
% Special for e-unibus doc commands

\newcommand{\ForLater}{
\begin{center}
{\bf NOT FOR CURRENT VERSION}
\end{center}
}
\newcommand{\TBC}{\framebox{\textbf{TO BE COMPLETED}}}
\newcommand{\DISCUSS}{\Ovalbox {\bf \textcolor{red}{FOR DISCUSSION}}}
\newcommand{\Input}{\framebox{\textsf{in}}}
\newcommand{\Output}{\framebox{\textsf{out}}}
\newcommand{\debug}[1]{\textbf{debug start} #1 \textbf{debug finish}}
\newcommand{\inx}[1]{\emph{#1}}
\newtheorem{notation}{Notation}
\newtheorem{definition}{Definition}
\newtheorem{problem_statement}{Problem Statement}
\newtheorem{invariant}{Invariant}
\newtheorem{assumption}{Assumption}
\newtheorem{resource_string}{Resource String}
\newtheorem{testcase}{Test Case}
\newtheorem{note}{Note}
\newtheorem{specification}{Specification}
\newtheorem{caution}{Caution}
\newtheorem{prereq}{Pre-requisite}
\newtheorem{action}{Action}
\newtheorem{query}{Query}
\newcommand{\beq}{\begin{equation}} %% new, no conflict
\newcommand{\eeq}{\end{equation}} %% new, no conflict
\newcommand{\be}{\begin{enumerate}}
\newcommand{\ee}{\end{enumerate}}
\newcommand{\bi}{\begin{itemize}}
\newcommand{\ei}{\end{itemize}}
\newcommand{\bv}{\begin{verbatim}}
\newcommand{\ev}{\end{verbatim}}
\newcommand{\bd}{\begin{description}}
\newcommand{\ed}{\end{description}}
\newcommand{\bpre}{\begin{prereq}}
\newcommand{\epre}{\end{prereq}}
\newcommand{\bact}{\begin{action}}
\newcommand{\eact}{\end{action}}
\newcommand{\bs}{\begin{specification}}
\newcommand{\es}{\end{specification}}
\newcommand{\btc}{\begin{testcase}}
\newcommand{\etc}{\end{testcase}}
\newcommand{\bc}{\begin{caution}}
\newcommand{\ec}{\end{caution}}
\newcommand{\la}{\leftarrow}
\newcommand{\IpArgs}{\subsection{Input Arguments}}
\newcommand{\PreReqs}{\subsection{Pre-requisities}}
\newcommand{\Actions}{\subsection{Actions}}
\newcommand{\Coverage}{{\bf To test coverage.}}

%%%%%%%%%%%%%%%%%%%%%%%%%%%%%%%%%%%%%%%%%%%%%%%%%%%%%%%%%%%%%%%%%%%%%%%%%%%


\newtheorem{theorem}{Theorem}[section]
\newtheorem{lemma}[theorem]{Lemma}
\newtheorem{proposition}[theorem]{Proposition}
\newtheorem{corollary}[theorem]{Corollary}

\newenvironment{proof}[1][Proof]{\begin{trivlist}
\item[\hskip \labelsep {\bfseries #1}]}{\end{trivlist}}
\newenvironment{intuition}[1][Intuition]{\begin{trivlist}
\item[\hskip \labelsep {\bfseries #1}]}{\end{trivlist}}
%% \newenvironment{definition}[1][Definition]{\begin{trivlist}
%% \item[\hskip \labelsep {\bfseries #1}]}{\end{trivlist}}
\newenvironment{example}[1][Example]{\begin{trivlist}
\item[\hskip \labelsep {\bfseries #1}]}{\end{trivlist}}
\newenvironment{remark}[1][Remark]{\begin{trivlist}
\item[\hskip \labelsep {\bfseries #1}]}{\end{trivlist}}

\newcommand{\qed}{\nobreak \ifvmode \relax \else
      \ifdim\lastskip<1.5em \hskip-\lastskip
      \hskip1.5em plus0em minus0.5em \fi \nobreak
      \vrule height0.75em width0.5em depth0.25em\fi}

%%%%%%%%%%%%%%%%%%%%%%%%%%%%%%%%%%%%%%%%%%%%%%%%%%%%%%%%%%%%%%%%%
% \newcommand{\Alogon}{\mbox{\fontfamily{ptm}\selectfont {\large \selectfont A} \hspace{-1.2ex} {\large \selectfont L} \hspace{-2.3ex} \raisebox{0.45ex}{ {\footnotesize \selectfont O} } \hspace{-1.80ex} {\large \selectfont G} \hspace{-1.80ex} \raisebox{-0.33ex}{ {\large \selectfont O} } \hspace{-1.8ex} {\large \selectfont N}}}




%%%%%%%%%%%%%%%%%%%%%%%%%%%%%%%%%%%%%%%%%%%%%%%%%%%%%%%%%%%%%%%%%%%%%%%%%%%%%%%%%%
% Report title page
% \newcommand{\ReportAuthor}{The LinkedIn Documentation Team}
\newcommand{\ReportAuthor}{Ramesh}
\newcommand{\reportauthor}[1]{\renewcommand{\ReportAuthor}{#1}}
\newcommand{\reporttitle}[1]{\renewcommand{\ReportTitle}{#1}}

\newcommand{\reporttitlepage}[1]{
  \begin{titlepage}
    {\noindent \raggedright\normalsize\sffamily \ReportAuthor}\par\vfill
    {\noindent \raggedright\Huge\sffamily \rule{\textwidth}{1pt}\\[5pt]#1\\[5pt]\rule{\textwidth}{1pt}}\par\vfill
    {\noindent \raggedright\normalsize\sffamily Version \docversion\hfill{}\docpublishingdate}\par\vfill
  \end{titlepage}
}
%%%%%%%%%%%%%%%%%%%%%%%%%%%%%%%%%%%%%%%%%%%%%%%%%%%%%%%%%%%%%%%%%%%%%%%%%%%%%%%%%%


\newcommand{\startreport}[1]
{
  \begin{document}
  \reporttitlepage{#1}
  \tableofcontents
  \specialchapter{#1}
}

\renewcommand{\thesection}{\arabic{section}}

\startreport{Decision Trees in Q}
\reportauthor{Ramesh Subramonian}

\section{Introduction}

\subsection{Notations}

\bi
\item Let \(F = \{f_i\}\) be a table of F4 vectors, representing the features.
\item Let \(g\) be a B1 vector, representing the outcome which we wish to
predict
\ei

A decision tree is a Lua table where each element identifies
\be
\item a feature
\item a threshold, the default comparison is always \(\leq\).
\item a left decision tree
\item a right decision tree
\ee

\begin{invariant}
\(forall f \in F, f:length() = g:length()\)
\end{invariant}

\begin{figure}
\centering
\fbox{
\begin{minipage}{35cm}
\begin{tabbing} \hspace*{0.25in} \=  \hspace*{0.25in} \=
                \hspace*{0.25in} \=  \hspace*{0.25in} \= \kill
Let \(\alpha\) be minimum benefit required to continue branching \\ 
Initialize, \(T = \{\}\) \\
\(F, g\) as described above \\
{\bf function } DT(T, F, g) \+ \ \\
  \(n_P, n = Q.sum(g) \) \\
  {\bf forall} \(f \in F:~ s(f), b(f) = \mathrm{Benefit}(f, g, n_N, n_P)\) \\ 
  Let \(f'\) be feature with maximum benefit \\
  {\bf if} benefit \(\geq \alpha\) {\bf then } \+  \\
    \(x = Q.vsgt(f', s(f'))\) \\
    \(n_R, n  = Q.sum(x) \) \\
    \(F_L = F_R = \{\}\) \\
    {\bf forall} \(f \in F\) {\bf do} \+ \\
      \(Q.reorder(f, x)\) \\ 
      \(F_L = F_L \cup Q.vector(f, 0, n_L)\) \\
      \(F_R = F_L \cup Q.vector(f, n_L, n)\) \- \\
    {\bf endfor} \\
    T.feature = \(f'\) \\
    T.threshold = \(b(f')\) \\
    T.left = \(\{\}\) \\ 
    T.right = \(\{\}\) \\ 
    \(DT(F_L, g_L, T_L)\) \\ 
    \(DT(F_R, g_R, T_R)\) \- \\ 
  {\bf endif} \- \\
{\bf end} 
\end{tabbing}
\end{minipage}
}
\label{dt_pseudo_code}
\caption{Decision Tree algorithm}
\end{figure}

\begin{figure}
\centering
\fbox{
\begin{minipage}{15cm}
\begin{tabbing} \hspace*{0.25in} \=  \hspace*{0.25in} \= 
                \hspace*{0.25in} \=  \hspace*{0.25in} \= \kill
{\bf function } \(\mathrm{Benefit}(f, g, n_N, n_P)\) \+  \\
  \(p_{max} = -\infty\) \\ 
  \(b_{opt} = \bot\) \\ 
  \(f', g' = \mathrm{Q.reorder}(f, g)\) \\
  counter = {}; counter[0] = 0; counter[1] = 0 \\
  idx = 0 \\ 
  n = f:length() \\ 
  REPEAT: \+ \\ 
  b = f[idx] \\
  counter[g[idx]]++ \\
  {\bf for } ( j = idx; \(j < n\); j++ ) {\bf do} \+ \\
    {\bf if } \(f_j \neq b\) {\bf then } \+ \\ 
      {\bf break} \- \\ 
    {\bf endif} \\ 
    counter[g[j]]++ \- \\
  {\bf endfor} \\ 
  \(p = \mathrm{WeightedBenefit}(counter[0], counter[1], n_N,n_P)\) \\
  {\bf if } \(p > p_{max}\) {\bf then } \+ \\ 
    \(p_{max} = p \)  \\
    \(b_{bot} = b \) \- \\
  {\bf endif} \\ 
  idx = j \\
  {\bf goto} REPEAT \- \\ 
  DONE \- \\
{\bf end} 
\end{tabbing}
\end{minipage}
}
\label{compute_benefit_numeric}
\caption{Benefit Computation (numeric attributes)}
\end{figure}
%%-------------------------------------------
\begin{figure}
\centering
\fbox{
\begin{minipage}{15cm}
\begin{tabbing} \hspace*{0.25in} \=  \hspace*{0.25in} \= \kill
{\bf function } \(\mathrm{WeightedBenefit}(n_N^L, n_P^L, n_N, n_P)\) \+  \\
  \(n_N^R = n_N - n_N^L\) \\
  \(n_P^R = n_P - n_P^L\) \\
  \(n_R = n_N^R  + n_P^R\) \\ 
  \(n_L = n_N^L  + n_P^L\) \\ 
  {\bf return} \( \frac{n_L}{n} \times XXX + \frac{n_R}{n} \times YYY \) \- \\
{\bf end} 
\end{tabbing}
\end{minipage}
}
\label{compute_weighted_benefit}
\caption{Weighted Benefit Computation}
\end{figure}
%%-------------------------------------------
\begin{figure}
\centering
\fbox{
\begin{minipage}{15cm}
\begin{tabbing} \hspace*{0.25in} \=  \hspace*{0.25in} \= \kill
{\bf function } \(\mathrm{Benefit}(f, g, n_N, n_P)\) \+ 

{\bf end} 
\end{tabbing}
\end{minipage}
}
\label{compute_benefit_boolean}
\caption{Benefit Computation (boolean attributes)}
\end{figure}
%%-------------------------------------------

\begin{figure}
\centering
\fbox{
\begin{minipage}{15cm}
\begin{tabbing} \hspace*{0.25in} \=  \hspace*{0.25in} \= \kill
{\bf function } \(\mathrm{Benefit}(f, g, n_N, n_P)\) \+ 

{\bf end} 
\end{tabbing}
\end{minipage}
}
\label{compute_benefit_categorical}
\caption{Benefit Computation (categorical attributes)}
\end{figure}
\end{document}
