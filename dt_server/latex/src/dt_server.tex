\startreport{The Random Forest Server}
\reportauthor{Ramesh Subramonian}

\section{Introduction}
\TBC

\subsection{Current Limitations}

\be
\item 
We do not support categorical attributes, 
only numeric and boolean attributes. 
\item We do not support unknown values. 
\ee

\subsection{What do you need to provided to launch a model?}

To deploy a model, 
\be
\item you need to create a directory that is accessible to the server. 
\item the directory name should contain only alphanumeric characters 
and the underscore character
\item The directory should contain the following files
\be
\item \verb+dt.csv+, Section~\ref{dt_csv}
\item \verb+mdl_meta.lua+, Section~\ref{mdl_meta}
\item \verb+dt_features.lua  +, Section~\ref{dt_features}
\item \verb+mdl_map.lua+, Section~\ref{mdl_map}
\item \verb+sample_input.json+ and \verb+sample_output.json+, Section~\ref{test_files}
\ee
\ee
We will now describe how you go about creating these files.

\section{Step by Step Guide}

\subsection{Build Random Forest using sklearn}
Assume that you have used sklearn to build your random forest.
\be
\item ``Pickle'' the model. Call this {\tt MODEL.pkl}.  MODEL.pkl should 
either 
\be
\item be a sklearn tree-based model, or 
\item a dictionary of: \begin{verbatim}
{model_name: sklearn tree-based model}
\end{verbatim}
\ee
\item Convert it into a format that the server likes:
\be
\item \verb+virtualenv venv+
\item \verb+source venv/bin/activate+
\item \verb+pip install -u upgrade pip+
\item \verb+pip install -r requirements.txt+
\item \verb+python convert_dt.py MODEL.pkl dt.csv mdl_map.lua+
\item \verb+deactivate+
\item \verb+rm -rf venv+
\ee
\ee

This will generate three files
\be
\item \verb+dt.csv+, Section~\ref{dt_csv}
\item \verb+mdl_map.lua+ , Section~\ref{mdl_map}
\item \verb+dt_features.lua+ , Section~\ref{dt_features}
\ee

\subsubsection{The Model CSV File}
\label{dt_csv}

\begin{definition}
We define a vector, \(X\) be ``sequential'' if \(X_0 = 0\) and
\(X_i = X_{i-1}\) or \(X_i = X_{i-1}+1\)
\end{definition}

The file \verb+dt.csv+ is a CSV file.
Types of the various fields are defined in Table~\ref{valid_types}. Note that
you {\bf cannot} provide 1.0 if the value is expected to be an integer.

\begin{table}
\centering
\begin{tabular}{|l|l|l|l|} \hline \hline
{\bf ID} & {\bf Field } & {\bf Type} & {\bf Checks} \\ \hline
1 & \verb+model_idx+ & \verb+int32_t+ & sequential \\ \hline
2 & \verb+tree_idx+  & \verb+int32_t+ & sequential \\ \hline
3 & \verb+node_idx+  & \verb+int32_t+ & sequential \\ \hline
4 & \verb+lchild_idx+  & \verb+int32_t+ & \\ \hline
5 & \verb+rchild_idx+  & \verb+int32_t+ & \\ \hline
6 & \verb+feature_idx+  & \verb+int32_t+ & \\ \hline
7 & \verb+threshold+  & \verb+float+ \\ \hline
8 & \verb+nneg+  & \verb+int32_t+ & non-negative integer\\ \hline
9 & \verb+npos+  & \verb+int32_t+ & non-negative integer\\ \hline
10 & \verb+xgb_val+  & \verb+float+ & \\ \hline
\end{tabular}
\label{tbl_data_sets}
\caption{Data Sets used for experimentation}
\end{table}

There are a lot more invariants to check. \TBC

\subsubsection{The Model Map File}
\label{mdl_map}
The model map file lists the various models for which random forests 
have been produced. It is a Lua table indexed as 1, 2, \ldots As an example,
\begin{verbatim}
return {[1]='model', [2] = 'some_other_model', [3] = 'yet_another_model'}
\end{verbatim}

\subsubsection{The Features for the Decision Tree}
\label{dt_features}

The input to the DT server is a JSON string e.g.,
\begin{verbatim}
{ "age" : 30, "height" : 175, "bp" : 120, "weight" = 210,  }
\end{verbatim}
The DT server maintains an array of floats, the size of which is equal to the
number of features. In this case, we would have \verb+float X[4]+ in C.
We need to consistently map each feature to a unique index in the C array. This is done
by the \verb+dt_features.lua+ file looking like 
\begin{verbatim}
return {[1]='age', [2] = 'height', [3] = 'weight', [4] = 'bp', }
\end{verbatim}

\subsubsection{Meta Data for Model}
\label{mdl_meta}
The file \verb+mdl_meta.lua+ is optional. If provided, it should be 
a set of key-value pairs, where both keys and values are strings. 

\subsubsection{Files for testing}
\label{test_files}
The data scientist should create at least one sample input for which the output,
produced by the sklearn package, is known. These help provide a basic sanity
test. 

